\documentclass[paperwidth=60in,paperheight=40in,landscape,fontscale=0.30]{baposter}

\usepackage{times}
\usepackage{calc}
\usepackage{graphicx}
\usepackage{amsmath}
\usepackage{amssymb}
\usepackage{relsize}
\usepackage{multirow}
\usepackage{bm}

\usepackage{graphicx}
\usepackage{multicol}
\usepackage{wrapfig}
\usepackage[font={footnotesize,it},labelfont={bf,up}]{caption}

\usepackage{pgfbaselayers}
\pgfdeclarelayer{background}
\pgfdeclarelayer{foreground}
\pgfsetlayers{background,main,foreground}

\usepackage{helvet}
%\usepackage{bookman}
\usepackage{palatino}

%\newcommand{\captionfont}{\small}

\selectcolormodel{rgb}

\graphicspath{{images/}}

%%%%%%%%%%%%%%%%%%%%%%%%%%%%%%%%%%%%%%%%%%%%%%%%%%%%%%%%%%%%%%%%%%%%%%%%%%%%%%%%
%%%% Some math symbols used in the text
%%%%%%%%%%%%%%%%%%%%%%%%%%%%%%%%%%%%%%%%%%%%%%%%%%%%%%%%%%%%%%%%%%%%%%%%%%%%%%%%
% Format 
\newcommand{\Matrix}[1]{\begin{bmatrix} #1 \end{bmatrix}}
\newcommand{\Vector}[1]{\Matrix{#1}}
\newcommand*{\SET}[1]  {\ensuremath{\mathcal{#1}}}
\newcommand*{\MAT}[1]  {\ensuremath{\mathbf{#1}}}
\newcommand*{\VEC}[1]  {\ensuremath{\bm{#1}}}
\newcommand*{\CONST}[1]{\ensuremath{\mathit{#1}}}
\newcommand*{\norm}[1]{\mathopen\| #1 \mathclose\|}% use instead of $\|x\|$
\newcommand*{\abs}[1]{\mathopen| #1 \mathclose|}% use instead of $\|x\|$
\newcommand*{\absLR}[1]{\left| #1 \right|}% use instead of $\|x\|$

\def\norm#1{\mathopen\| #1 \mathclose\|}% use instead of $\|x\|$
\newcommand{\normLR}[1]{\left\| #1 \right\|}% use instead of $\|x\|$

%%%%%%%%%%%%%%%%%%%%%%%%%%%%%%%%%%%%%%%%%%%%%%%%%%%%%%%%%%%%%%%%%%%%%%%%%%%%%%%%
% Multicol Settings
%%%%%%%%%%%%%%%%%%%%%%%%%%%%%%%%%%%%%%%%%%%%%%%%%%%%%%%%%%%%%%%%%%%%%%%%%%%%%%%%
\setlength{\columnsep}{0.7em}
\setlength{\columnseprule}{0mm}
%\setlength{\parindent}{2em}


%%%%%%%%%%%%%%%%%%%%%%%%%%%%%%%%%%%%%%%%%%%%%%%%%%%%%%%%%%%%%%%%%%%%%%%%%%%%%%%%
% Save space in lists. Use this after the opening of the list
%%%%%%%%%%%%%%%%%%%%%%%%%%%%%%%%%%%%%%%%%%%%%%%%%%%%%%%%%%%%%%%%%%%%%%%%%%%%%%%%
\newcommand{\compresslist}{%
\setlength{\itemsep}{1pt}%
\setlength{\parskip}{0pt}%
\setlength{\parsep}{0pt}%
}


%%%%%%%%%%%%%%%%%%%%%%%%%%%%%%%%%%%%%%%%%%%%%%%%%%%%%%%%%%%%%%%%%%%%%%%%%%%%%%
%%% Begin of Document
%%%%%%%%%%%%%%%%%%%%%%%%%%%%%%%%%%%%%%%%%%%%%%%%%%%%%%%%%%%%%%%%%%%%%%%%%%%%%%

\begin{document}

%%%%%%%%%%%%%%%%%%%%%%%%%%%%%%%%%%%%%%%%%%%%%%%%%%%%%%%%%%%%%%%%%%%%%%%%%%%%%%
%%% Here starts the poster
%%%---------------------------------------------------------------------------
%%% Format it to your taste with the options
%%%%%%%%%%%%%%%%%%%%%%%%%%%%%%%%%%%%%%%%%%%%%%%%%%%%%%%%%%%%%%%%%%%%%%%%%%%%%%
\typeout{Poster Starts}
\background{
  \begin{tikzpicture}[remember picture,overlay]%
    \draw (current page.north west)+(-2em,-0em) node[anchor=north west] {\hspace{-2em}\includegraphics[height=1.1\textheight]{silhouettes_background}};
  \end{tikzpicture}%
}
\definecolor{white}{rgb}{1,1,1}
\definecolor{lightblue}{rgb}{0.835,0.9,0.937}
\definecolor{darkblue}{rgb}{0.2,0.2,0.7}
\begin{poster}{
  % Show grid to help with alignment
  grid=false,
  columns=3,
  % Column spacing
  colspacing=1em,
  % Color style
  bgColorOne=white,
  bgColorTwo=white,
  borderColor=darkblue,
  headerColorOne=darkblue,
  headerColorTwo=darkblue,
  headerFontColor=white,
  boxColorOne=lightblue,
  boxColorTwo=lightblue,
  % Format of textbox
  textborder=rounded,
  % Format of text header
  eyecatcher=false,
  headerborder=open,
  headerheight=0.08\textheight,  % title size
  headershape=rounded,
  headershade=plain,
  headerfont=\LARGE\textsf, %Sans Serif
  boxshade=plain,
  textfont={\setlength{\parindent}{1.2em}},
%  background=shade-tb,
  background=plain,
  linewidth=2pt,
  }
  % Eye Catcher
  {} % No eye catcher for this poster. If an eye catcher is present, the title is centered between eye-catcher and logo.
  % Title
  {\sf %Sans Serif
  %\bf% Serif
	 Automated Earthquake Surface-Wave Phase Velocity Measurement of USArray
  }
  % Authors
  {\sf %Sans Serif
  \\
	{\LARGE Ge Jin, James Gaherty}
  }
  % University logo
  {
\setlength\fboxsep{0pt}
\setlength\fboxrule{0.0pt}
	\fbox{
		\begin{minipage}{40em}
			\includegraphics[width=40em]{pics/ldeologo}
		\end{minipage}
	}
  }

  \tikzstyle{light shaded}=[top color=baposterBGtwo!30!white,bottom color=baposterBGone!30!white,shading=axis,shading angle=30]

  % Width of left inset image
     \newlength{\leftimgwidth}
     \setlength{\leftimgwidth}{0.78em+8.0em}

%%%%%%%%%%%%%%%%%%%%%%%%%%%%%%%%%%%%%%%%%%%%%%%%%%%%%%%%%%%%%%%%%%%%%%%%%%%%%%
%%% Now define the boxes that make up the poster
%%%---------------------------------------------------------------------------
%%% Each box has a name and can be placed absolutely or relatively.
%%% The only inconvenience is that you can only specify a relative position 
%%% towards an already declared box. So if you have a box attached to the 
%%% bottom, one to the top and a third one which should be in between, you 
%%% have to specify the top and bottom boxes before you specify the middle 
%%% box.
%%%%%%%%%%%%%%%%%%%%%%%%%%%%%%%%%%%%%%%%%%%%%%%%%%%%%%%%%%%%%%%%%%%%%%%%%%%%%%
    %
    % A coloured circle useful as a bullet with an adjustably strong filling
    \newcommand{\colouredcircle}[1]{%
      \tikz{\useasboundingbox (-0.2em,-0.32em) rectangle(0.2em,0.32em); \draw[draw=black,fill=baposterBGone!80!black!#1!white,line width=0.03em] (0,0) circle(0.18em);}}

%%%%%%%%%%%%%%%%%%%%%%%%%%%%%%%%%%%%%%%%%%%%%%%%%%%%%%%%%%%%%%%%%%%%%%%%%%%%%%
  \headerbox{Abstract}{name=abstract,column=0,row=0,span=1}{
%%%%%%%%%%%%%%%%%%%%%%%%%%%%%%%%%%%%%%%%%%%%%%%%%%%%%%%%%%%%%%%%%%%%%%%%%%%%%%
Earthquake surface waves are recorded by dense seismic arrays as a strong and consistent signal among the stations. By taking advantage of the wavefield similarity of nearby stations, we have developed a new technique to estimate surface wave phase velocities precisely and automatically. 
First, a time window to isolate fundamental-mode surface wave energy is built based on the group delays of all frequency bands of interest. We then calculate multi-channel broadband cross-correlation functions of the isolated waveforms from nearby stations, and fit narrow-band filtered cross correlations  with a five-parameter controlled wavelet to retrieve the optimized phase difference at a range of frequencies. The amplitude of this cross-correlation function can be used to estimate the coherence, which together with signal to noise ratio (SNR) are the two key factors to exclude unqualified measurements. 
The phase difference information between all the nearby station pairs for each event at each frequency is then used as the input to an Eikonal tomographic inversion for two-dimensional estimates of apparent phase velocity. We measure the amplitude of each station by adapting the same process on the auto-correlation function and perform Helmholtz tomography to estimate and remove interference (focusing and defocusing) effects. Finally, we stack the measurements of each individual event to get the final phase-velocity tomogram in each frequency band.
The entire process requires no human interaction and is highly automated.  This method can be applied on data from arrays of various apertures, ranging from continental scale such as USArray, to regional scales (few hundred km) typical ofPASSCAL arrays, to local-scale high-frequency arrays employed in industry and hazard investigations. By combining this analysis with programmable data acquisition services such as SOD or IRIS DMC web services, we can easily set up an automatic system providing up-to-date surface wave phase velocity information of USArray to the public.
We apply this method to data from USArray and a small temporary PASSCAL array in Papua New Guinea. The results provide important constraints on the crustal and upper mantle structure of these regions. The figure attached shows the Rayleigh and Love wave phase velocity tomograms from USArray at the frequency bands centering around 25s and 50s.


\vspace{2ex}
 }


%%%%%%%%%%%%%%%%%%%%%%%%%%%%%%%%%%%%%%%%%%%%%%%%%%%%%%%%%%%%%%%%%%%%%%%%%%%%%%
  \headerbox{Introduction}{name=introduction,column=0,span=1,below=abstract,above=bottom}{
%%%%%%%%%%%%%%%%%%%%%%%%%%%%%%%%%%%%%%%%%%%%%%%%%%%%%%%%%%%%%%%%%%%%%%%%%%%%%%
  }

%%%%%%%%%%%%%%%%%%%%%%%%%%%%%%%%%%%%%%%%%%%%%%%%%%%%%%%%%%%%%%%%%%%%%%%%%%%%%%
  \headerbox{Data Analysis}{name=data,column=1,span=1,row=0}{
%%%%%%%%%%%%%%%%%%%%%%%%%%%%%%%%%%%%%%%%%%%%%%%%%%%%%%%%%%%%%%%%%%%%%%%%%%%%%%


  }

%%%%%%%%%%%%%%%%%%%%%%%%%%%%%%%%%%%%%%%%%%%%%%%%%%%%%%%%%%%%%%%%%%%%%%%%%%%%%%
  \headerbox{Rayleigh Wave Phase Velocity}{name=results,column=2,span=1,row=0}{
%%%%%%%%%%%%%%%%%%%%%%%%%%%%%%%%%%%%%%%%%%%%%%%%%%%%%%%%%%%%%%%%%%%%%%%%%%%%%%
  }

\end{poster}%
%
\end{document}
